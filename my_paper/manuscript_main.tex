%%%%%%%%%%%%%%%%%%%%%%%%%%%%%%%%%%%
% Main Text
%%%%%%%%%%%%%%%%%%%%%%%%%%%%%%%%%%%
\section{Introduction}\label{sec:intro}

Arabs have been a part of the United States since the 19th century, and the Arab American population has grown significantly in the past few decades. However, the Arab American population is understudied, and there is a lack of data on this population.This paper aims to fill this gap by providing a more accurate method of estimating the number of Arabs in the United States. 

I use the Current Population Survey (CPS) to estimate the number of Arab Americans in the United States. The CPS is a monthly survey of households conducted by the U.S. Census Bureau. The survey collects information on a variety of topics, including employment, income, and education. The CPS is a nationally representative survey, and it is widely used by researchers to study the U.S. population. To count Arabs, I use a method developed by \textcite{antmanEthnicAttritionObserved2016} to identify four generation+ of Hispanics in the US. I apply this method to the Arab population in the CPS to estimate the number of Arab Americans in the United States. By linking children 17 years and younger to their parents, I can observe three generations of Arab Americans in the CPS. I then use this information to estimate the number of Arab Americans in the United States. I also examine the degree of underrepresentation of Arabs in the economic profession. I scrape the names of all economists in the US from the top 25 economics departments and use a machine learning algorithm to classify the names as Arab or non-Arab. 

I find that Arabs are extremely undercounted in the US. In the 2020 Census, Arab American population is estimated to be 2.8 million. This count, however, is highly likely to be an underestimate since Arabs do not have an ethnic or racial box to identify themselves. Therefore, researchers rely on the place of birth to identify Arabs. This undercounting has important implications for public policy and research. For example, the undercounting of Arabs in the US could lead to a misallocation of resources and an inaccurate representation of the Arab American population. By providing a more accurate estimate of the number of Arab Americans in the United States, this paper can help policymakers and researchers better understand the Arab American population and develop policies that meet their needs.

\section{Data}\label{sec:data}

I use the IPUMS Current Population Survey (CPS) from 1994 to 2023 \autocite{floodsarahIntegratedPublicUse2021a}. The CPS data (1994-2023) is used to study the effect of Hispanic-sounding last name on labor market outcomes. 

\subsection{Counting Three Generations of Arabs}\label{subsec:cps}

I measure Arabic identity using the Current Population Survey (CPS), which allows me to construct an objective measure of the Arabic identity of minors living with their parents. I will use the information on the place of birth, parent's place of birth, and place of birth of grandparents to construct an objective Arabic measure.\footnote{Following the works of \textcite{antmanEthnicAttritionObserved2016}.} Thus, I could perfectly identify and construct a dataset of first-, second-, and third-generation Arab immigrants. This will consequently allow me to build an objective measure of the Arab identity of first- and second-generation Arabs and third-generation Arab minors under the age 17 living with their parents. 
\subsection{Data Cleaning and Validation}\label{subsec:data_cleaning}

To ensure the reliability of the analysis, the CPS data undergoes a rigorous cleaning process. Missing values in key variables such as place of birth and generation status are addressed through imputation methods where appropriate. Additionally, consistency checks are performed to verify the accuracy of generational classifications. Outliers in socioeconomic indicators are identified and examined to prevent skewed results. This meticulous data preparation facilitates a robust estimation of the Arab American population and the subsequent analysis of labor market outcomes.

\section{Results}\label{sec:results}

\subsection{Demographic Characteristics}\label{subsec:demographic_characteristics}

The estimated Arab American population exhibits a diverse demographic profile. The majority are concentrated in urban areas with significant metropolitan populations. Educational attainment levels among Arab Americans are comparable to the national average, with a substantial proportion holding bachelor's and advanced degrees. Employment rates indicate a stable labor market presence, with key industries including healthcare, education, and technology.

\subsection{Labor Market Outcomes}\label{subsec:labor_market_outcomes}

Analysis of labor market outcomes reveals disparities in employment sectors and income levels among Arab Americans. The regression models indicate that generational status significantly influences employment probability and income, with second and third-generation Arab Americans experiencing higher employment rates and income levels compared to first-generation immigrants. Educational attainment is a strong predictor of labor market success, highlighting the role of education in economic assimilation.

\section{Discussion}\label{sec:discussion}

The findings underscore the undercounting of Arab Americans in national surveys and the implications for public policy and resource allocation. The generational differences in labor market outcomes suggest pathways for economic integration and highlight the need for targeted support in education and employment sectors. Limitations of the study include potential misclassification due to reliance on place of birth and the constraints of survey data in capturing nuanced ethnic identities.

\section{Conclusion}\label{sec:conclusion}

This study provides a more accurate estimation of the Arab American population in the United States by leveraging generational data from the Current Population Survey. The results demonstrate significant undercounting in official census data and reveal important trends in demographic and economic integration. Policymakers and researchers can utilize these findings to inform strategies that address the needs of the Arab American community, ensuring equitable representation and resource distribution.

